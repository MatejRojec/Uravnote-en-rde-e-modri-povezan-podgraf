\documentclass[a4paper, 11pt]{article}
\usepackage[slovene]{babel}
\usepackage[utf8]{inputenc}
\usepackage{amsthm}
\usepackage{amsmath}
\usepackage{amssymb}
\usepackage{graphicx}
\usepackage{relsize}


\theoremstyle{definition}
\newtheorem{definicija}{Definicija}
\newtheorem{izrek}{Izrek}
\newtheorem{trditev}{Trditev}
\newtheorem{lema}{Lema}
\newtheorem{primer}{Primer}

\newcommand{\G}{\mathcal{G}}
\newcommand{\E}{\mathcal{E}}
\newcommand{\V}{\mathcal{V}}

\begin{document}
    \begin{titlepage}
        \begin{center}
            \vspace*{1cm}
            
            \LARGE
            FINANČNI PRAKTIKU
                
            \vspace{1cm}
            \huge
            \textbf{Uravnotežen rdeče-modri povezan podgraf}
                
            \vspace{1cm}
                
            \Large   
            Matej Rojec, Ana Marija Belingar, Vito Rozman 
        
            
            \vspace{2cm}
    
            %\includegraphics[scale=0.7]{logo_fmf_uni-lj_sl.pdf}
                
                    
        \end{center}
    \end{titlepage}

    %\tableofcontents
    %\listoftables
    \section{Predstavitev problema}

    Naj bo $G = (V, E)$ graf. Vsako vozlišče $v \in V $ je obarvano rdeče ali modro. 
    Najti želimo največji povezani podgraf $G' = (V', E')$, ki ima enako število rdečih in modrih vozlišč.
    Velikost podgrafa je število njegovih vozlišč. Ta problem je v splošnem NP-težek, kar pomeni da ga ne moremo
    rešiti v polinomskem času.\\
    Osredotočili se bomo na optimalen algoritem za reševanje problema na mrežah oblike $1 \times n$ (pot), $2 \times n$, 
    $3 \times n$ in $4 \times n$.\\
    Naš algoritrm bomo testirali na grafih, kjer bomo vožlišča obarvali rdeče z verjetnostjo $p \in (0,1)$ 
    in modro z verjetnostjo $1 - p$. 
    
    \subsection{Osnovni pojmi}
    
    \begin{definicija}
        Naj bo $G=(V,E)$ graf in naj bo $S \subseteq V$ podmnožica vozlišča grafa $G$. 
        Graf $G[S]$ je induciran podgraf grafa $G$, natanko takrat ko $\forall u, v \in S$ velja,
        da sta $u$ in $v$ sosednja v $G[S]$, natanko takrat ko sta sosednja v $G$.
    \end{definicija}

    V našem primeru bomo iskali tak inducirani podgraf $G' = (V', E')$ grafa $G = (V, E)$, kjer lahko možico volzlišč
    zapišemo kot 
        $$ V' = V_{R} \cup V_{B},$$
    za katero velja $ V_{R} \cap  V_{B} = \emptyset $ in $|V_{R}| = |V_{B}| = \frac{|V'|}{2}$. Tako bomo dobil
    uravnotežen povezan graf z enako števlo rdečih in modrih vozlišč.
    %V našem problemu 
    %proučujemo graf $G=(V,E)$, ki ima vsako vozlišče obarvano bodisi rdeče bodisi modro. 
    %(Opomba: barva dodelitev ne ustreza samo dvodelnim grafom. Torej dopustimo tudi, 
    %da sta sosednji vozlišči iste barve.) Iščemo največjo podmnožico $V'\subseteq V$, ki je barvno uravnotežena 
    %(ima $\frac{|V'|}{2}$ rdečih in $\frac{|V'|}{2}$ modrih oglišč) in taka, da je inducirani podgraf 
    %$G[V'$s] povezan. To imenujemo problem največjega uravnoteženega povezanega podgrafa.


  %V našem problemu 
    %proučujemo graf $G=(V,E)$, ki ima vsako vozlišče obarvano bodisi rdeče bodisi modro. 
    %(Opomba: barva dodelitev ne ustreza samo dvodelnim grafom. Torej dopustimo tudi, 
    %da sta sosednji vozlišči iste barve.) Iščemo največjo podmnožico $V'\subseteq V$, ki je barvno uravnotežena 
    %(ima $\frac{|V'|}{2}$ rdečih in $\frac{|V'|}{2}$ modrih oglišč) in taka, da je inducirani podgraf 
    %$G[V'$s] povezan. To imenujemo problem največjega uravnoteženega povezanega podgrafa.

    % Vito, namesto navadni G, V, E uporabi raje tiste iz \mathcal. Definiral sem ti bližnjice \G, \V, \E. lp Oskar

    \section{Ideje}

    Najprej si bomo ogledali problem na poteh ($1 \times n $). Ugotovili smo, da se da ta problem rešiti 
    na "brut force" način, torej brez uporabe dinamičnega programiranja.\\
    Definerajmo $ G_{i,j} :=$ največji uteženi podgraf od $i$-tega do $j$-tega vozlšča,\\
    $u_{i} :=$ utež vozlišča $v_i$; $\forall u_{i} \in \{-1, 1\} $
    Psevdo koda:\\
    for $i= 1,\dots,n$\\
    for $j= i,\dots,n$\\
    if $ \mathlarger{\sum_{k=i}^{j}u_{k} = 0} $\\
    $ G_{i,j}  = \{v_i,\dots, v_j \}$ \\
    else $ G_{i,j}  = \emptyset$
    \\
    \\
    Rešitev : max($G_{i,j}$)















\end{document}  
