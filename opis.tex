\documentclass[a4paper, 11pt]{article}
\usepackage[slovene]{babel}
\usepackage[utf8]{inputenc}
\usepackage[T1]{fontenc}
\usepackage{amsfonts,amsmath,amssymb}
\usepackage{amsthm}
\usepackage{amsmath}
\usepackage{amssymb}
\usepackage{graphicx}
\usepackage{relsize}
\usepackage{algpseudocode}  % za psevdokodo
\usepackage{algorithm}      % za
\usepackage[all]{xy}

\floatname{algorithm}{Algoritem}
\renewcommand{\listalgorithmname}{Kazalo algoritmov}
\algnewcommand\algorithmicto{\textbf{to}}
\algnewcommand\algorithmicin{\textbf{in}}
\algnewcommand\algorithmicforeach{\textbf{for each}}
\algrenewtext{For}[3]{\algorithmicfor\ #1 $\gets$ #2\ \algorithmicto\ #3\ \algorithmicdo}
\algdef{S}[FOR]{ForEach}[2]{\algorithmicforeach\ #1\ \algorithmicin\ #2\ \algorithmicdo}


\theoremstyle{definition}
\newtheorem{definicija}{Definicija}
\theoremstyle{definition}
\newtheorem{opomba}{Opomba}
\newtheorem{izrek}{Izrek}
\newtheorem{trditev}{Trditev}
\newtheorem{lema}{Lema}
\newtheorem{primer}{Primer}

\newcommand{\G}{\mathcal{G}}
\newcommand{\E}{\mathcal{E}}
\newcommand{\V}{\mathcal{V}}

\begin{document}
    \begin{titlepage}
        \begin{center}
            \vspace*{1cm}
            
            \LARGE
            FINANČNI PRAKTIKU
                
            \vspace{1cm}
            \huge
            \textbf{Uravnotežen rdeče-modri povezan podgraf}
                
            \vspace{1cm}
                
            \Large   
            Matej Rojec, Ana Marija Belingar, Vito Rozman 
        
            
            \vspace{2cm}
    
            %\includegraphics[scale=0.7]{logo_fmf_uni-lj_sl.pdf}
                
                    
        \end{center}
    \end{titlepage}

    %\tableofcontents
    %\listoftables
    \section{Predstavitev problema}
        
    Naj bo $G = (V, E)$ graf. Vsako vozlišče $v \in V $ je obarvano rdeče ali modro. 
    Najti želimo največji povezani podgraf $G' = (V', E')$, ki ima enako število rdečih in modrih vozlišč.
    Velikost podgrafa je število njegovih vozlišč. Ta problem je v splošnem NP-težek, kar pomeni da ga ne moremo
    rešiti v polinomskem času.
    Osredotočili se bomo na optimalen algoritem za reševanje problema na mrežah oblike $1 \times n$ (pot), $2 \times n$, 
    $3 \times n$ in $4 \times n$.
    Naš algoritem bomo testirali na grafih, kjer bomo vožlišča obarvali rdeče z verjetnostjo $p \in (0,1)$ 
    in modro z verjetnostjo $1 - p$. 
    

    \section{Osnovni pojmi}
    
    \begin{definicija} (Induciran podgraf)
        Naj bo $G=(V,E)$ graf in naj bo $S \subseteq V$ podmnožica vozlišč grafa $G$. 
        Graf $G[S]$ je induciran podgraf grafa $G$, natanko takrat ko $\forall u, v \in S$ velja,
        da sta $u$ in $v$ sosednja v $G[S]$, natanko takrat ko sta sosednja v $G$.
    \end{definicija}
    
    V našem primeru bomo iskali največji povezan inducirani podgraf \\ $G' = (V', E')$ grafa $G = (V, E)$, kjer lahko možico volzlišč
    zapišemo kot: 
        $$ V' = V_{R} \cup V_{B},$$
    za katero velja $ V_{R} \cap  V_{B} = \emptyset $ in $|V_{R}| = |V_{B}| = \frac{|V'|}{2}$. Tako bomo dobili
    največji uravnotežen povezan podgraf z enakim številom rdečih in modrih vozlišč.    

    \begin{definicija} (Pot) Graf $P_n=(V,E)$ je pot, kjer je
        $V=\{ v_1,...,v_n \}$, če je množica povezav enaka:
        $E = \{ (v_1,v_2),(v_2,v_3),...,(v_{n-1},v_n) \}$.
    \end{definicija}

    \begin{definicija} 
    (Mreža) Graf $M_{m,n}=(V,E)$ je mreža velikosti $m \times n$, 
    kjer je $V=\{ v_{1,1},...,v_{1,n},v_{2,1},...,v_{2,n},...,v_{m,1},...,v_{m,n} \}$, če je množica povezav oblike:
    \begin{align*}
    E =\{ &(v_{1,1}, v_{1,2}),(v_{1,2},v_{1,3}),...,(v_{1,n-1},v_{1,n}),  \\
        &(v_{2,1}, v_{2,2}),(v_{2,2},v_{2,3}),...,(v_{2,n-1},v_{2,n}),  \\
        &(v_{m,1}, v_{m,2}),(v_{m,2},v_{m,3}),...,(v_{m,n-1},v_{1,n}), \\
        &(v_{1,1}, v_{2,1}),(v_{2,1},v_{3,1}),...,(v_{m-1,1},v_{m,1}),  \\
        &(v_{1,2}, v_{2,2}),(v_{2,2},v_{3,2}),...,(v_{m-1,2},v_{m,2}), \\
        &(v_{1,n}, v_{2,n}),(v_{2,n},v_{3,n}),...,(v_{m-1,n},v_{m,n})  \}.  
    \end{align*}
    \end{definicija}

    \section{Ideje}

    Najprej si bomo ogledali problem na poteh ($1 \times n $), pri čemer bomo uporabili
    dinamično programiranje kot je razvidno iz algoritma \ref{alg:BCS1} .    
    Algoritem sprejme kot vhod pot dolžine $n$, kjer je vsako vozlišče
    obarvano bodisi modro (B) bodisi rdeče (R). Najprej barvam dodelimo uteži
    $u_i$, modrim vozliščem dodelimo utež 1, rdečim pa utež -1. 
    Potem na vsakem koraku izračunamo delno vsoto $s_i$, ki pove razliko med
    številom modrih in rdečih vozlišč od prvega do $i-$ tega vozlišča.
    Nato izračunamo še $t_{j,i}$, ki pove razliko med
    številom modrih in rdečih vozlišč od $j-$tega do $i-$tega.  
    Kandidati so tisti $t_{j,i}$ za katere je $t_{j,i}=0$.
    Med temi vzamemo največje podpoti $P_{l,k}$, pri katerih je razlika
    $l-k$ maksimalna. 
\begin{algorithm}[ht]
  \caption{BCS za poti .}
  \label{alg:BCS1}
  \textbf{Vhod:} Pot $P_n$,
  , barve vozlišč (B/R).     \\ 
  \textbf{Izhod:} Najdaljša utežena (povezana) podpot $P_{l,k}$.  
  \begin{algorithmic}[1]
    \Function{BSCP}{$P_n$}
    \For{$i$}{$1$}{$n$}
        \If{Barva($v_i$) = B}
            \State    $u_i  \leftarrow 1$ 
        \Else
            \State $u_i \leftarrow -1$
        \EndIf
    \EndFor
    \State $s_0 \leftarrow 0$
    \For{$i$}{$1$}{$n$}
          \State $s_i \leftarrow s_{i-1} + u_i$
          \For{$j$}{$1$}{$i$}
            \State $t_{j,i} = s_i - s_{j-1}$
          \EndFor
    \EndFor

    \State $ len \leftarrow \max \{ i-j \mid t_{j,i}=0 \} $
    \State $P_{l,k} \leftarrow \{ \{ v_l,...,v_k \} \mid (t_{l,k} \leftarrow 0)~ \textnormal{\textbf{and}}~ k-l = len $ \}

    \State \Return $P_{l,k}$  \label{alg:pomembna vrstica}
    \EndFunction
  \end{algorithmic}
\end{algorithm}


\textbf{Načrt za naprej:}  
\begin{itemize}
    \item Posplošitev algoritma na mreže
    \item Implementacija algoritma
    \item Eksprerimenti 
\end{itemize} 












\end{document}  
